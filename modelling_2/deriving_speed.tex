\documentclass[]{article}
\usepackage{lmodern}
\usepackage{amssymb,amsmath}
\usepackage{ifxetex,ifluatex}
\usepackage{fixltx2e} % provides \textsubscript
\ifnum 0\ifxetex 1\fi\ifluatex 1\fi=0 % if pdftex
  \usepackage[T1]{fontenc}
  \usepackage[utf8]{inputenc}
\else % if luatex or xelatex
  \ifxetex
    \usepackage{mathspec}
  \else
    \usepackage{fontspec}
  \fi
  \defaultfontfeatures{Ligatures=TeX,Scale=MatchLowercase}
\fi
% use upquote if available, for straight quotes in verbatim environments
\IfFileExists{upquote.sty}{\usepackage{upquote}}{}
% use microtype if available
\IfFileExists{microtype.sty}{%
\usepackage{microtype}
\UseMicrotypeSet[protrusion]{basicmath} % disable protrusion for tt fonts
}{}
\usepackage[margin=1in]{geometry}
\usepackage{hyperref}
\hypersetup{unicode=true,
            pdftitle={Deriving variables: Speed},
            pdfauthor={Don Li},
            pdfborder={0 0 0},
            breaklinks=true}
\urlstyle{same}  % don't use monospace font for urls
\usepackage{color}
\usepackage{fancyvrb}
\newcommand{\VerbBar}{|}
\newcommand{\VERB}{\Verb[commandchars=\\\{\}]}
\DefineVerbatimEnvironment{Highlighting}{Verbatim}{commandchars=\\\{\}}
% Add ',fontsize=\small' for more characters per line
\usepackage{framed}
\definecolor{shadecolor}{RGB}{248,248,248}
\newenvironment{Shaded}{\begin{snugshade}}{\end{snugshade}}
\newcommand{\AlertTok}[1]{\textcolor[rgb]{0.94,0.16,0.16}{#1}}
\newcommand{\AnnotationTok}[1]{\textcolor[rgb]{0.56,0.35,0.01}{\textbf{\textit{#1}}}}
\newcommand{\AttributeTok}[1]{\textcolor[rgb]{0.77,0.63,0.00}{#1}}
\newcommand{\BaseNTok}[1]{\textcolor[rgb]{0.00,0.00,0.81}{#1}}
\newcommand{\BuiltInTok}[1]{#1}
\newcommand{\CharTok}[1]{\textcolor[rgb]{0.31,0.60,0.02}{#1}}
\newcommand{\CommentTok}[1]{\textcolor[rgb]{0.56,0.35,0.01}{\textit{#1}}}
\newcommand{\CommentVarTok}[1]{\textcolor[rgb]{0.56,0.35,0.01}{\textbf{\textit{#1}}}}
\newcommand{\ConstantTok}[1]{\textcolor[rgb]{0.00,0.00,0.00}{#1}}
\newcommand{\ControlFlowTok}[1]{\textcolor[rgb]{0.13,0.29,0.53}{\textbf{#1}}}
\newcommand{\DataTypeTok}[1]{\textcolor[rgb]{0.13,0.29,0.53}{#1}}
\newcommand{\DecValTok}[1]{\textcolor[rgb]{0.00,0.00,0.81}{#1}}
\newcommand{\DocumentationTok}[1]{\textcolor[rgb]{0.56,0.35,0.01}{\textbf{\textit{#1}}}}
\newcommand{\ErrorTok}[1]{\textcolor[rgb]{0.64,0.00,0.00}{\textbf{#1}}}
\newcommand{\ExtensionTok}[1]{#1}
\newcommand{\FloatTok}[1]{\textcolor[rgb]{0.00,0.00,0.81}{#1}}
\newcommand{\FunctionTok}[1]{\textcolor[rgb]{0.00,0.00,0.00}{#1}}
\newcommand{\ImportTok}[1]{#1}
\newcommand{\InformationTok}[1]{\textcolor[rgb]{0.56,0.35,0.01}{\textbf{\textit{#1}}}}
\newcommand{\KeywordTok}[1]{\textcolor[rgb]{0.13,0.29,0.53}{\textbf{#1}}}
\newcommand{\NormalTok}[1]{#1}
\newcommand{\OperatorTok}[1]{\textcolor[rgb]{0.81,0.36,0.00}{\textbf{#1}}}
\newcommand{\OtherTok}[1]{\textcolor[rgb]{0.56,0.35,0.01}{#1}}
\newcommand{\PreprocessorTok}[1]{\textcolor[rgb]{0.56,0.35,0.01}{\textit{#1}}}
\newcommand{\RegionMarkerTok}[1]{#1}
\newcommand{\SpecialCharTok}[1]{\textcolor[rgb]{0.00,0.00,0.00}{#1}}
\newcommand{\SpecialStringTok}[1]{\textcolor[rgb]{0.31,0.60,0.02}{#1}}
\newcommand{\StringTok}[1]{\textcolor[rgb]{0.31,0.60,0.02}{#1}}
\newcommand{\VariableTok}[1]{\textcolor[rgb]{0.00,0.00,0.00}{#1}}
\newcommand{\VerbatimStringTok}[1]{\textcolor[rgb]{0.31,0.60,0.02}{#1}}
\newcommand{\WarningTok}[1]{\textcolor[rgb]{0.56,0.35,0.01}{\textbf{\textit{#1}}}}
\usepackage{graphicx}
% grffile has become a legacy package: https://ctan.org/pkg/grffile
\IfFileExists{grffile.sty}{%
\usepackage{grffile}
}{}
\makeatletter
\def\maxwidth{\ifdim\Gin@nat@width>\linewidth\linewidth\else\Gin@nat@width\fi}
\def\maxheight{\ifdim\Gin@nat@height>\textheight\textheight\else\Gin@nat@height\fi}
\makeatother
% Scale images if necessary, so that they will not overflow the page
% margins by default, and it is still possible to overwrite the defaults
% using explicit options in \includegraphics[width, height, ...]{}
\setkeys{Gin}{width=\maxwidth,height=\maxheight,keepaspectratio}
\IfFileExists{parskip.sty}{%
\usepackage{parskip}
}{% else
\setlength{\parindent}{0pt}
\setlength{\parskip}{6pt plus 2pt minus 1pt}
}
\setlength{\emergencystretch}{3em}  % prevent overfull lines
\providecommand{\tightlist}{%
  \setlength{\itemsep}{0pt}\setlength{\parskip}{0pt}}
\setcounter{secnumdepth}{0}
% Redefines (sub)paragraphs to behave more like sections
\ifx\paragraph\undefined\else
\let\oldparagraph\paragraph
\renewcommand{\paragraph}[1]{\oldparagraph{#1}\mbox{}}
\fi
\ifx\subparagraph\undefined\else
\let\oldsubparagraph\subparagraph
\renewcommand{\subparagraph}[1]{\oldsubparagraph{#1}\mbox{}}
\fi

%%% Use protect on footnotes to avoid problems with footnotes in titles
\let\rmarkdownfootnote\footnote%
\def\footnote{\protect\rmarkdownfootnote}

%%% Change title format to be more compact
\usepackage{titling}

% Create subtitle command for use in maketitle
\providecommand{\subtitle}[1]{
  \posttitle{
    \begin{center}\large#1\end{center}
    }
}

\setlength{\droptitle}{-2em}

  \title{Deriving variables: Speed}
    \pretitle{\vspace{\droptitle}\centering\huge}
  \posttitle{\par}
    \author{Don Li}
    \preauthor{\centering\large\emph}
  \postauthor{\par}
      \predate{\centering\large\emph}
  \postdate{\par}
    \date{05/06/2020}


\begin{document}
\maketitle

\hypertarget{data-and-stuff}{%
\section{Data and stuff}\label{data-and-stuff}}

This follows on from the imputation of distance.

\begin{Shaded}
\begin{Highlighting}[]
\KeywordTok{load}\NormalTok{( }\StringTok{"G:/azure_hackathon/datasets2/model_subset/testing_subset1_imputedist.RData"}\NormalTok{ )}
\KeywordTok{load}\NormalTok{( }\StringTok{"G:/azure_hackathon/datasets2/model_subset/testing_historicals.RData"}\NormalTok{ )}
\NormalTok{historical_vars =}\StringTok{ }\KeywordTok{colnames}\NormalTok{( historicals )[}
    \KeywordTok{grepl}\NormalTok{( }\StringTok{"historic"}\NormalTok{, }\KeywordTok{colnames}\NormalTok{( historicals ) )}
\NormalTok{    ]}
\NormalTok{training_set[ historicals, }\KeywordTok{eval}\NormalTok{(historical_vars) }\OperatorTok{:}\ErrorTok{=}\StringTok{ }\NormalTok{\{}
    \KeywordTok{list}\NormalTok{( i.historic_timediff,}
\NormalTok{        i.historic_crow_dist,}
\NormalTok{        i.historic_path_dist,}
\NormalTok{        i.historic_sampling_rate,}
\NormalTok{        i.historic_mean_speed,}
\NormalTok{        i.historic_log_var_speed}
\NormalTok{    )}
\NormalTok{\}, on =}\StringTok{ }\KeywordTok{c}\NormalTok{(}\StringTok{"weekday"}\NormalTok{, }\StringTok{"hour"}\NormalTok{) ]}
\NormalTok{test_set[ historicals, }\KeywordTok{eval}\NormalTok{(historical_vars) }\OperatorTok{:}\ErrorTok{=}\StringTok{ }\NormalTok{\{}
    \KeywordTok{list}\NormalTok{( i.historic_timediff,}
\NormalTok{        i.historic_crow_dist,}
\NormalTok{        i.historic_path_dist,}
\NormalTok{        i.historic_sampling_rate,}
\NormalTok{        i.historic_mean_speed,}
\NormalTok{        i.historic_log_var_speed}
\NormalTok{    )}
\NormalTok{\}, on =}\StringTok{ }\KeywordTok{c}\NormalTok{(}\StringTok{"weekday"}\NormalTok{, }\StringTok{"hour"}\NormalTok{) ]}

\NormalTok{vars_to_rm =}\StringTok{ }\KeywordTok{c}\NormalTok{(}\StringTok{"trj_id"}\NormalTok{, }\StringTok{"timediff"}\NormalTok{, }\StringTok{"sampling_rate"}\NormalTok{,}
    \StringTok{"sampling_rate_var"}\NormalTok{, }\StringTok{"var_speed"}\NormalTok{ )}
\NormalTok{training_set[ , }\KeywordTok{eval}\NormalTok{(vars_to_rm) }\OperatorTok{:}\ErrorTok{=}\StringTok{ }\OtherTok{NULL}\NormalTok{ ]}
\end{Highlighting}
\end{Shaded}

\hypertarget{what-are-we-doing}{%
\section{What are we doing?}\label{what-are-we-doing}}

In this document, we want to consider predicting average journey speed.
You will note that in the model inputs, speed is not a given input:

\begin{itemize}
\tightlist
\item
  latitude\_origin
\item
  longitude\_origin
\item
  latitude\_destination
\item
  longitude\_destination
\item
  hour\_of\_day
\item
  day\_of\_week
\end{itemize}

Clearly, speed is a useful variable if we want to predict ETA. Speed is
included in the trip summaries as \texttt{mean\_pt\_speed} and
\texttt{var\_pt\_speed}. The speed is computed by taking the distance
between GPS pings and dividing by the time between them.

We will first impute \texttt{mean\_pt\_speed} and then
\texttt{var\_pd\_speed}.

\hypertarget{linear-regression}{%
\section{Linear regression}\label{linear-regression}}

\begin{Shaded}
\begin{Highlighting}[]
\NormalTok{yvar =}\StringTok{ "mean_speed"}
\NormalTok{xvars =}\StringTok{ }\KeywordTok{setdiff}\NormalTok{( }\KeywordTok{names}\NormalTok{(training_set), yvar )}
\KeywordTok{par}\NormalTok{( }\DataTypeTok{mfrow =} \KeywordTok{c}\NormalTok{(}\DecValTok{5}\NormalTok{,}\DecValTok{4}\NormalTok{ ) )}
\NormalTok{col  =}\StringTok{ }\KeywordTok{rgb}\NormalTok{( }\DecValTok{0}\NormalTok{, }\DecValTok{0}\NormalTok{, }\DecValTok{0}\NormalTok{, }\FloatTok{0.25}\NormalTok{ )}
\ControlFlowTok{for}\NormalTok{( x }\ControlFlowTok{in}\NormalTok{ xvars )\{}
    \KeywordTok{plot}\NormalTok{( training_set[[x]], training_set[[yvar]],}
        \DataTypeTok{pch =} \DecValTok{16}\NormalTok{, }\DataTypeTok{col =}\NormalTok{ col, }\DataTypeTok{ylab =}\NormalTok{ yvar,}
        \DataTypeTok{xlab =}\NormalTok{ x, }\DataTypeTok{main =}\NormalTok{ x )}
    \KeywordTok{lines}\NormalTok{( }\KeywordTok{smooth.spline}\NormalTok{( training_set[[x]], training_set[[yvar]] ),}
        \DataTypeTok{col =} \StringTok{"red"}\NormalTok{)}
\NormalTok{\}}

\NormalTok{model_formula_lm =}\StringTok{ }\NormalTok{mean_speed }\OperatorTok{~}\StringTok{ }\NormalTok{. }\OperatorTok{+}
\StringTok{    }\NormalTok{rush_hour }\OperatorTok{*}\StringTok{ }\KeywordTok{I}\NormalTok{(crow_dist}\OperatorTok{^}\DecValTok{2}\NormalTok{) }\OperatorTok{+}
\StringTok{    }\NormalTok{rush_hour }\OperatorTok{*}\StringTok{ }\KeywordTok{I}\NormalTok{(path_dist_impute}\OperatorTok{^}\DecValTok{2}\NormalTok{) }\OperatorTok{+}
\StringTok{    }\KeywordTok{I}\NormalTok{(hour}\OperatorTok{^}\DecValTok{2}\NormalTok{) }\OperatorTok{+}\StringTok{ }\KeywordTok{I}\NormalTok{(hour}\OperatorTok{^}\DecValTok{3}\NormalTok{) }\OperatorTok{+}
\StringTok{    }\KeywordTok{I}\NormalTok{(start_y}\OperatorTok{^}\DecValTok{2}\NormalTok{) }\OperatorTok{+}\StringTok{ }\KeywordTok{I}\NormalTok{(end_y}\OperatorTok{^}\DecValTok{2}\NormalTok{) }\OperatorTok{+}
\StringTok{    }\KeywordTok{I}\NormalTok{( azure_dist}\OperatorTok{^}\DecValTok{2}\NormalTok{ ) }\OperatorTok{+}\StringTok{ }\KeywordTok{I}\NormalTok{( OSRM_dist}\OperatorTok{^}\DecValTok{2}\NormalTok{ ) }\OperatorTok{+}
\StringTok{    }\KeywordTok{I}\NormalTok{( path_dist_impute}\OperatorTok{^}\DecValTok{2}\NormalTok{ ) }\OperatorTok{+}\StringTok{ }\KeywordTok{I}\NormalTok{(path_dist2_impute}\OperatorTok{^}\DecValTok{2}\NormalTok{) }\OperatorTok{+}
\StringTok{    }\NormalTok{crow_dist}\OperatorTok{:}\NormalTok{azure_dist }\OperatorTok{+}\StringTok{ }\NormalTok{crow_dist}\OperatorTok{:}\NormalTok{OSRM_dist }\OperatorTok{+}\StringTok{ }\NormalTok{crow_dist}\OperatorTok{:}\NormalTok{path_dist_impute }\OperatorTok{+}
\StringTok{    }\NormalTok{crow_dist}\OperatorTok{:}\NormalTok{historic_path_dist }\OperatorTok{+}\StringTok{ }\NormalTok{crow_dist}\OperatorTok{:}\NormalTok{historic_crow_dist }\OperatorTok{+}\StringTok{ }\NormalTok{crow_dist}\OperatorTok{:}\NormalTok{historic_sampling_rate }\OperatorTok{+}
\StringTok{    }\NormalTok{weekday}\OperatorTok{:}\NormalTok{start_y }\OperatorTok{+}\StringTok{ }\NormalTok{weekday}\OperatorTok{:}\NormalTok{start_x }\OperatorTok{+}\StringTok{ }\NormalTok{weekday}\OperatorTok{:}\NormalTok{end_x }\OperatorTok{+}\StringTok{ }\NormalTok{weekday}\OperatorTok{:}\NormalTok{end_y }\OperatorTok{+}
\StringTok{    }\NormalTok{weekday}\OperatorTok{:}\NormalTok{path_dist_impute }\OperatorTok{+}\StringTok{ }\NormalTok{weekday}\OperatorTok{:}\NormalTok{azure_dist }\OperatorTok{+}\StringTok{ }\NormalTok{weekday}\OperatorTok{:}\NormalTok{OSRM_dist }\OperatorTok{+}
\StringTok{    }\NormalTok{hour }\OperatorTok{*}\StringTok{ }\NormalTok{rush_hour }\OperatorTok{+}
\StringTok{    }\NormalTok{rush_hour}\OperatorTok{:}\NormalTok{start_y }\OperatorTok{+}\StringTok{ }\NormalTok{rush_hour}\OperatorTok{:}\NormalTok{start_x }\OperatorTok{+}\StringTok{ }\NormalTok{rush_hour}\OperatorTok{:}\NormalTok{end_x }\OperatorTok{+}\StringTok{ }\NormalTok{rush_hour}\OperatorTok{:}\NormalTok{end_y}

\NormalTok{lm_ =}\StringTok{ }\KeywordTok{train}\NormalTok{( }\DataTypeTok{form =}\NormalTok{ enet_formula,}
    \DataTypeTok{data =}\NormalTok{ training_set,}
    \DataTypeTok{metric =} \StringTok{"RMSE"}\NormalTok{, }\DataTypeTok{method =} \StringTok{"lm"}\NormalTok{, }\DataTypeTok{trControl =}\NormalTok{ train_control)}

\NormalTok{lm_results =}\StringTok{ }\KeywordTok{data.table}\NormalTok{( lm_}\OperatorTok{$}\NormalTok{results )}
\NormalTok{lm_pred =}\StringTok{ }\KeywordTok{data.table}\NormalTok{( lm_}\OperatorTok{$}\NormalTok{pred )}
\KeywordTok{setorder}\NormalTok{( lm_pred, rowIndex )}

\KeywordTok{save}\NormalTok{( lm_, lm_results, lm_pred, }
    \DataTypeTok{file =} \StringTok{"G:/azure_hackathon/datasets2/expo_speed/lm.RData"}\NormalTok{ )}
\end{Highlighting}
\end{Shaded}

\begin{Shaded}
\begin{Highlighting}[]
\KeywordTok{load}\NormalTok{( }\StringTok{"G:/azure_hackathon/datasets2/expo_speed/lm.RData"}\NormalTok{ )}
\NormalTok{lm_results}
\end{Highlighting}
\end{Shaded}

\begin{verbatim}
##    intercept        RMSE Rsquared         MAE      RMSESD RsquaredSD
## 1:      TRUE 0.002202268 0.661809 0.001736866 6.01315e-05 0.02385513
##           MAESD
## 1: 3.484545e-05
\end{verbatim}

\hypertarget{elastic-net}{%
\section{Elastic net}\label{elastic-net}}

Elastic net. No interactions in the enet because it is very slow.

\begin{Shaded}
\begin{Highlighting}[]
\NormalTok{n_enet =}\StringTok{ }\DecValTok{50}
\NormalTok{enet_tunegrid =}\StringTok{ }\KeywordTok{data.frame}\NormalTok{(}
    \DataTypeTok{lambda =} \KeywordTok{rexp}\NormalTok{( n_enet, }\DecValTok{1}\OperatorTok{/}\FloatTok{0.000000001}\NormalTok{ ),}
    \DataTypeTok{fraction =} \KeywordTok{runif}\NormalTok{( n_enet, }\FloatTok{0.75}\NormalTok{, }\DecValTok{1}\NormalTok{ )}
\NormalTok{)}
\NormalTok{enet_ =}\StringTok{ }\KeywordTok{train}\NormalTok{( }\DataTypeTok{form =}\NormalTok{ model_formula_lm, }\DataTypeTok{data =}\NormalTok{ training_set,}
    \DataTypeTok{metric =} \StringTok{"RMSE"}\NormalTok{, }\DataTypeTok{method =} \StringTok{"enet"}\NormalTok{, }\DataTypeTok{trControl =}\NormalTok{ train_control,}
    \DataTypeTok{tuneGrid =}\NormalTok{ enet_tunegrid, }\DataTypeTok{standardize =} \OtherTok{TRUE}\NormalTok{, }\DataTypeTok{intercept =} \OtherTok{TRUE}
\NormalTok{    )}

\NormalTok{enet_results =}\StringTok{ }\KeywordTok{data.table}\NormalTok{( enet_}\OperatorTok{$}\NormalTok{results )}
\NormalTok{enet_pred =}\StringTok{ }\KeywordTok{data.table}\NormalTok{( enet_}\OperatorTok{$}\NormalTok{pred )}
\KeywordTok{setorder}\NormalTok{( enet_pred, rowIndex )}

\KeywordTok{save}\NormalTok{( enet_, enet_results, enet_pred, }
    \DataTypeTok{file =} \StringTok{"G:/azure_hackathon/datasets2/expo_speed/enet.RData"}\NormalTok{ )}
\end{Highlighting}
\end{Shaded}

\begin{Shaded}
\begin{Highlighting}[]
\KeywordTok{load}\NormalTok{( }\StringTok{"G:/azure_hackathon/datasets2/expo_speed/enet.RData"}\NormalTok{ )}
\NormalTok{enet_results[ }\KeywordTok{which.min}\NormalTok{(RMSE) ]}
\end{Highlighting}
\end{Shaded}

\begin{verbatim}
##          lambda fraction        RMSE  Rsquared         MAE       RMSESD
## 1: 1.343621e-11 0.816212 0.002188906 0.6661094 0.001723912 5.486761e-05
##    RsquaredSD        MAESD
## 1: 0.01917431 4.423906e-05
\end{verbatim}

\begin{Shaded}
\begin{Highlighting}[]
\KeywordTok{par}\NormalTok{( }\DataTypeTok{mfrow =} \KeywordTok{c}\NormalTok{(}\DecValTok{2}\NormalTok{, }\DecValTok{1}\NormalTok{ ) )}
\NormalTok{enet_results[ , \{}
    \KeywordTok{plot}\NormalTok{( fraction, RMSE, }\DataTypeTok{type =} \StringTok{"o"}\NormalTok{ )}
\NormalTok{    nu_order =}\StringTok{ }\KeywordTok{order}\NormalTok{(lambda)}
    \KeywordTok{plot}\NormalTok{( lambda[nu_order], RMSE[nu_order], }\DataTypeTok{type =} \StringTok{"o"}\NormalTok{ )}
\NormalTok{\} ]}
\end{Highlighting}
\end{Shaded}

\includegraphics{deriving_speed_files/figure-latex/unnamed-chunk-5-1.pdf}

\begin{verbatim}
## NULL
\end{verbatim}

\hypertarget{partial-least-squares}{%
\section{Partial least squares}\label{partial-least-squares}}

PLS.

\begin{Shaded}
\begin{Highlighting}[]
\NormalTok{full_X =}\StringTok{ }\KeywordTok{ncol}\NormalTok{( }\KeywordTok{model.matrix}\NormalTok{( model_formula_lm, training_set) )}

\NormalTok{pls_tunegrid =}\StringTok{ }\KeywordTok{data.frame}\NormalTok{( }\DataTypeTok{ncomp =} \DecValTok{1}\OperatorTok{:}\NormalTok{full_X )}

\NormalTok{pls_ =}\StringTok{ }\KeywordTok{train}\NormalTok{( }\DataTypeTok{form =}\NormalTok{ model_formula_lm, }\DataTypeTok{data =}\NormalTok{ training_set,}
    \DataTypeTok{metric =} \StringTok{"RMSE"}\NormalTok{, }\DataTypeTok{method =} \StringTok{"pls"}\NormalTok{, }\DataTypeTok{trControl =}\NormalTok{ train_control,}
    \DataTypeTok{tuneGrid =}\NormalTok{ pls_tunegrid, }\DataTypeTok{scale =}\NormalTok{ T}
\NormalTok{    )}

\NormalTok{pls_results =}\StringTok{ }\KeywordTok{data.table}\NormalTok{( pls_}\OperatorTok{$}\NormalTok{results )}
\NormalTok{pls_pred =}\StringTok{ }\KeywordTok{data.table}\NormalTok{( pls_}\OperatorTok{$}\NormalTok{pred )}
\KeywordTok{setorder}\NormalTok{( pls_pred, rowIndex )}

\KeywordTok{save}\NormalTok{( pls_, pls_results, pls_pred, }
    \DataTypeTok{file =} \StringTok{"G:/azure_hackathon/datasets2/expo_speed/pls.RData"}\NormalTok{ )}
\end{Highlighting}
\end{Shaded}

\begin{Shaded}
\begin{Highlighting}[]
\KeywordTok{load}\NormalTok{( }\StringTok{"G:/azure_hackathon/datasets2/expo_speed/pls.RData"}\NormalTok{ )}
\NormalTok{pls_results[ }\KeywordTok{which.min}\NormalTok{(RMSE) ]}
\end{Highlighting}
\end{Shaded}

\begin{verbatim}
##    ncomp        RMSE  Rsquared        MAE       RMSESD RsquaredSD        MAESD
## 1:   110 0.002189175 0.6660742 0.00172433 5.485825e-05 0.01917639 4.421795e-05
\end{verbatim}

\begin{Shaded}
\begin{Highlighting}[]
\NormalTok{pls_results[ RMSE }\OperatorTok{<}\StringTok{ }\FloatTok{0.005}\NormalTok{, \{}
    \KeywordTok{plot}\NormalTok{( ncomp, RMSE, }\DataTypeTok{type =} \StringTok{"o"}\NormalTok{ )}
\NormalTok{\} ]}
\end{Highlighting}
\end{Shaded}

\includegraphics{deriving_speed_files/figure-latex/unnamed-chunk-7-1.pdf}

\begin{verbatim}
## NULL
\end{verbatim}

\hypertarget{principal-components-regression}{%
\section{Principal components
regression}\label{principal-components-regression}}

PCR.

\begin{Shaded}
\begin{Highlighting}[]
\NormalTok{full_X =}\StringTok{ }\KeywordTok{ncol}\NormalTok{( }\KeywordTok{model.matrix}\NormalTok{( model_formula_lm, training_set) )}
\NormalTok{pcr_tunegrid =}\StringTok{ }\KeywordTok{data.frame}\NormalTok{( }\DataTypeTok{ncomp =} \DecValTok{1}\OperatorTok{:}\NormalTok{full_X )}
\NormalTok{pcr_ =}\StringTok{ }\KeywordTok{train}\NormalTok{( }\DataTypeTok{form =}\NormalTok{ model_formula_lm, }\DataTypeTok{data =}\NormalTok{ training_set,}
    \DataTypeTok{metric =} \StringTok{"RMSE"}\NormalTok{, }\DataTypeTok{method =} \StringTok{"pcr"}\NormalTok{, }\DataTypeTok{trControl =}\NormalTok{ train_control,}
    \DataTypeTok{tuneGrid =}\NormalTok{ pcr_tunegrid, }\DataTypeTok{scale =}\NormalTok{ T}
\NormalTok{    )}

\NormalTok{pcr_results =}\StringTok{ }\KeywordTok{data.table}\NormalTok{( pcr_}\OperatorTok{$}\NormalTok{results )}
\NormalTok{pcr_pred =}\StringTok{ }\KeywordTok{data.table}\NormalTok{( pcr_}\OperatorTok{$}\NormalTok{pred )}
\KeywordTok{setorder}\NormalTok{( pcr_pred, rowIndex )}

\KeywordTok{save}\NormalTok{( pcr_, pcr_results, pcr_pred, }
    \DataTypeTok{file =} \StringTok{"G:/azure_hackathon/datasets2/expo_speed/pcr.RData"}\NormalTok{ )}
\end{Highlighting}
\end{Shaded}

\begin{Shaded}
\begin{Highlighting}[]
\KeywordTok{load}\NormalTok{( }\StringTok{"G:/azure_hackathon/datasets2/expo_speed/pcr.RData"}\NormalTok{ )}
\NormalTok{pcr_results[ }\KeywordTok{which.min}\NormalTok{(RMSE) ]}
\end{Highlighting}
\end{Shaded}

\begin{verbatim}
##    ncomp        RMSE  Rsquared         MAE       RMSESD RsquaredSD        MAESD
## 1:   116 0.002189038 0.6661158 0.001724253 5.475017e-05 0.01917707 4.407067e-05
\end{verbatim}

\begin{Shaded}
\begin{Highlighting}[]
\NormalTok{pcr_results[ RMSE }\OperatorTok{<}\StringTok{ }\FloatTok{0.0025}\NormalTok{, \{}
    \KeywordTok{plot}\NormalTok{( ncomp, RMSE, }\DataTypeTok{type =} \StringTok{"o"}\NormalTok{ )}
\NormalTok{\} ]}
\end{Highlighting}
\end{Shaded}

\includegraphics{deriving_speed_files/figure-latex/unnamed-chunk-9-1.pdf}

\begin{verbatim}
## NULL
\end{verbatim}

\hypertarget{knn}{%
\section{KNN}\label{knn}}

KNN.

\begin{Shaded}
\begin{Highlighting}[]
\NormalTok{k_grid =}\StringTok{ }\KeywordTok{data.frame}\NormalTok{( }\DataTypeTok{k =} \DecValTok{15}\OperatorTok{:}\DecValTok{50}\NormalTok{ )}

\NormalTok{knn_ =}\StringTok{ }\KeywordTok{train}\NormalTok{( }\DataTypeTok{form =}\NormalTok{ model_formula_lm, }\DataTypeTok{data =}\NormalTok{ training_set,}
    \DataTypeTok{metric =} \StringTok{"RMSE"}\NormalTok{, }\DataTypeTok{method =} \StringTok{"knn"}\NormalTok{, }\DataTypeTok{trControl =}\NormalTok{ train_control,}
    \DataTypeTok{tuneGrid =}\NormalTok{ k_grid}
\NormalTok{    )}

\NormalTok{knn_results =}\StringTok{ }\KeywordTok{data.table}\NormalTok{( knn_}\OperatorTok{$}\NormalTok{results )}
\NormalTok{knn_pred =}\StringTok{ }\KeywordTok{data.table}\NormalTok{( knn_}\OperatorTok{$}\NormalTok{pred )}
\KeywordTok{setorder}\NormalTok{( knn_pred, rowIndex )}

\KeywordTok{save}\NormalTok{( knn_, knn_results, knn_pred,}
    \DataTypeTok{file =} \StringTok{"G:/azure_hackathon/datasets2/expo_speed/knn.RData"}\NormalTok{ )}
\end{Highlighting}
\end{Shaded}

\begin{Shaded}
\begin{Highlighting}[]
\KeywordTok{load}\NormalTok{( }\StringTok{"G:/azure_hackathon/datasets2/expo_speed/knn.RData"}\NormalTok{ )}
\NormalTok{knn_results[ }\KeywordTok{which.min}\NormalTok{(RMSE) ]}
\end{Highlighting}
\end{Shaded}

\begin{verbatim}
##     k       RMSE  Rsquared         MAE       RMSESD RsquaredSD        MAESD
## 1: 18 0.00236419 0.6114566 0.001871393 5.432699e-05 0.02493647 3.363607e-05
\end{verbatim}

\begin{Shaded}
\begin{Highlighting}[]
\NormalTok{knn_results[ , }\KeywordTok{plot}\NormalTok{( k, RMSE, }\DataTypeTok{type =} \StringTok{"o"}\NormalTok{ ) ]}
\end{Highlighting}
\end{Shaded}

\includegraphics{deriving_speed_files/figure-latex/unnamed-chunk-11-1.pdf}

\begin{verbatim}
## NULL
\end{verbatim}

\hypertarget{cart}{%
\section{CART}\label{cart}}

Use an Exponential distribution for our random search.

\begin{Shaded}
\begin{Highlighting}[]
\NormalTok{cp_grid =}\StringTok{ }\KeywordTok{data.frame}\NormalTok{( }\DataTypeTok{cp =} \KeywordTok{rexp}\NormalTok{( }\DecValTok{100}\NormalTok{, }\DecValTok{1}\OperatorTok{/}\FloatTok{0.001}\NormalTok{ ) )}

\NormalTok{rpart_ =}\StringTok{ }\KeywordTok{train}\NormalTok{( mean_speed }\OperatorTok{~}\StringTok{ }\NormalTok{., }
    \DataTypeTok{data =}\NormalTok{ training_set,}
    \DataTypeTok{metric =} \StringTok{"RMSE"}\NormalTok{, }\DataTypeTok{method =} \StringTok{"rpart"}\NormalTok{, }\DataTypeTok{trControl =}\NormalTok{ train_control,}
    \DataTypeTok{tuneGrid =}\NormalTok{ cp_grid}
\NormalTok{    )}

\NormalTok{rpart_results =}\StringTok{ }\KeywordTok{data.table}\NormalTok{( rpart_}\OperatorTok{$}\NormalTok{results )}
\NormalTok{rpart_pred =}\StringTok{ }\KeywordTok{data.table}\NormalTok{( rpart_}\OperatorTok{$}\NormalTok{pred )}
\KeywordTok{setorder}\NormalTok{( rpart_pred, rowIndex )}

\KeywordTok{save}\NormalTok{( rpart_, rpart_results, rpart_pred,}
    \DataTypeTok{file =} \StringTok{"G:/azure_hackathon/datasets2/expo_speed/rpart.RData"}\NormalTok{ )}
\end{Highlighting}
\end{Shaded}

\begin{Shaded}
\begin{Highlighting}[]
\KeywordTok{load}\NormalTok{( }\StringTok{"G:/azure_hackathon/datasets2/expo_speed/rpart.RData"}\NormalTok{ )}
\NormalTok{rpart_results[ }\KeywordTok{which.min}\NormalTok{(RMSE) ]}
\end{Highlighting}
\end{Shaded}

\begin{verbatim}
##             cp       RMSE  Rsquared         MAE       RMSESD RsquaredSD
## 1: 0.001618599 0.00236741 0.6098749 0.001870525 7.435699e-05 0.02326198
##           MAESD
## 1: 5.990607e-05
\end{verbatim}

\begin{Shaded}
\begin{Highlighting}[]
\KeywordTok{plot}\NormalTok{( rpart_results}\OperatorTok{$}\NormalTok{cp, rpart_results}\OperatorTok{$}\NormalTok{RMSE, }\DataTypeTok{type =} \StringTok{"l"}\NormalTok{ )}
\end{Highlighting}
\end{Shaded}

\includegraphics{deriving_speed_files/figure-latex/unnamed-chunk-13-1.pdf}

\hypertarget{gam-splines}{%
\section{GAM splines}\label{gam-splines}}

Generalised additive model using splines

\begin{Shaded}
\begin{Highlighting}[]
\NormalTok{gam_grid =}\StringTok{ }\KeywordTok{expand.grid}\NormalTok{(}
    \DataTypeTok{select =}\NormalTok{ F,}
    \DataTypeTok{method =} \KeywordTok{c}\NormalTok{( }\StringTok{"GACV.Cp"}\NormalTok{ )}
\NormalTok{)}

\NormalTok{gam_ =}\StringTok{ }\KeywordTok{train}\NormalTok{( mean_speed }\OperatorTok{~}\StringTok{ }\NormalTok{., }\DataTypeTok{data =}\NormalTok{ training_set,}
    \DataTypeTok{metric =} \StringTok{"RMSE"}\NormalTok{, }\DataTypeTok{method =} \StringTok{"gam"}\NormalTok{, }\DataTypeTok{trControl =}\NormalTok{ train_control,}
    \DataTypeTok{tuneGrid =}\NormalTok{ gam_grid}
\NormalTok{)}

\NormalTok{gam_results =}\StringTok{ }\KeywordTok{data.table}\NormalTok{( gam_}\OperatorTok{$}\NormalTok{results )}
\NormalTok{gam_pred =}\StringTok{ }\KeywordTok{data.table}\NormalTok{( gam_}\OperatorTok{$}\NormalTok{pred )}
\KeywordTok{setorder}\NormalTok{( gam_pred, rowIndex )}

\KeywordTok{save}\NormalTok{( gam_, gam_results, gam_pred,}
    \DataTypeTok{file =} \StringTok{"G:/azure_hackathon/datasets2/expo_speed/gam_spline.RData"}\NormalTok{ )}
\end{Highlighting}
\end{Shaded}

\begin{Shaded}
\begin{Highlighting}[]
\KeywordTok{load}\NormalTok{( }\StringTok{"G:/azure_hackathon/datasets2/expo_speed/gam_spline.RData"}\NormalTok{ )}
\NormalTok{gam_results[ }\KeywordTok{which.min}\NormalTok{(RMSE) ]}
\end{Highlighting}
\end{Shaded}

\begin{verbatim}
##    select  method        RMSE  Rsquared         MAE       RMSESD RsquaredSD
## 1:  FALSE GACV.Cp 0.002163538 0.6736461 0.001710183 5.788122e-05 0.02018379
##           MAESD
## 1: 3.965227e-05
\end{verbatim}

\hypertarget{stacking}{%
\section{Stacking}\label{stacking}}

\begin{Shaded}
\begin{Highlighting}[]
\NormalTok{training_OOF =}\StringTok{ }\KeywordTok{data.table}\NormalTok{(}
    \DataTypeTok{lm =}\NormalTok{ lm_pred}\OperatorTok{$}\NormalTok{pred,}
    \DataTypeTok{pls =}\NormalTok{ pls_pred}\OperatorTok{$}\NormalTok{pred,}
    \DataTypeTok{pcr =}\NormalTok{ pcr_pred}\OperatorTok{$}\NormalTok{pred,}
    \DataTypeTok{knn =}\NormalTok{ knn_pred}\OperatorTok{$}\NormalTok{pred, }
    \DataTypeTok{rpart =}\NormalTok{ rpart_pred}\OperatorTok{$}\NormalTok{pred,}
    \DataTypeTok{gam =}\NormalTok{ gam_pred}\OperatorTok{$}\NormalTok{pred,}
    \DataTypeTok{mean_speed =}\NormalTok{ training_set}\OperatorTok{$}\NormalTok{mean_speed}
\NormalTok{)}

\NormalTok{test_OOF =}\StringTok{ }\KeywordTok{data.table}\NormalTok{(}
    \DataTypeTok{lm =} \KeywordTok{predict}\NormalTok{( lm_, test_set ),}
    \DataTypeTok{pls =} \KeywordTok{predict}\NormalTok{( pls_, test_set ),}
    \DataTypeTok{pcr =} \KeywordTok{predict}\NormalTok{( pcr_, test_set ),}
    \DataTypeTok{knn =} \KeywordTok{predict}\NormalTok{( knn_, test_set ),}
    \DataTypeTok{rpart =} \KeywordTok{predict}\NormalTok{( rpart_, test_set ),}
    \DataTypeTok{gam =} \KeywordTok{predict}\NormalTok{( gam_, test_set ),}
    \DataTypeTok{mean_speed =}\NormalTok{ test_set}\OperatorTok{$}\NormalTok{mean_speed}
\NormalTok{)}

\NormalTok{stacking_model_formula =}\StringTok{ }\KeywordTok{as.formula}\NormalTok{( }\StringTok{"mean_speed ~ ."}\NormalTok{ )}

\NormalTok{stacked_tunegrid =}\StringTok{ }\KeywordTok{data.frame}\NormalTok{( }\DataTypeTok{ncomp =} \DecValTok{1}\OperatorTok{:}\NormalTok{(}\KeywordTok{ncol}\NormalTok{(training_OOF)}\OperatorTok{-}\DecValTok{1}\NormalTok{) )}
\NormalTok{stacking =}\StringTok{ }\KeywordTok{train}\NormalTok{( stacking_model_formula, }\DataTypeTok{data =}\NormalTok{ training_OOF,}
    \DataTypeTok{metric =} \StringTok{"RMSE"}\NormalTok{, }\DataTypeTok{method =} \StringTok{"pls"}\NormalTok{, }\DataTypeTok{trControl =}\NormalTok{ train_control,}
    \DataTypeTok{tuneGrid =}\NormalTok{ stacked_tunegrid}
\NormalTok{    )}

\NormalTok{stacking_results =}\StringTok{ }\KeywordTok{data.table}\NormalTok{( stacking}\OperatorTok{$}\NormalTok{results )}
\NormalTok{stack_pred_OOF =}\StringTok{ }\KeywordTok{predict}\NormalTok{( stacking, test_OOF )}

\KeywordTok{save}\NormalTok{( stacking_results, stack_pred_OOF, test_OOF,}
    \DataTypeTok{file =} \StringTok{"G:/azure_hackathon/datasets2/expo_speed/stack.RData"}\NormalTok{ )}
\end{Highlighting}
\end{Shaded}

\begin{Shaded}
\begin{Highlighting}[]
\KeywordTok{load}\NormalTok{( }\StringTok{"G:/azure_hackathon/datasets2/expo_speed/stack.RData"}\NormalTok{ )}

\NormalTok{stacked_rmse =}\StringTok{ }\KeywordTok{sqrt}\NormalTok{( }\KeywordTok{mean}\NormalTok{( ( stack_pred_OOF }\OperatorTok{-}\StringTok{ }\NormalTok{test_OOF}\OperatorTok{$}\NormalTok{mean_speed )}\OperatorTok{^}\DecValTok{2}\NormalTok{ ) )}

\NormalTok{all_rmse =}\StringTok{ }\KeywordTok{rbind}\NormalTok{(}
    \KeywordTok{as.matrix}\NormalTok{(}\KeywordTok{sqrt}\NormalTok{( }\KeywordTok{colMeans}\NormalTok{( ( test_OOF }\OperatorTok{-}\StringTok{ }\NormalTok{test_OOF}\OperatorTok{$}\NormalTok{mean_speed )}\OperatorTok{^}\DecValTok{2}\NormalTok{ ) )),}
    \DataTypeTok{stack =}\NormalTok{ stacked_rmse}
\NormalTok{)}

\NormalTok{all_rmse[ }\KeywordTok{order}\NormalTok{(all_rmse), ]}
\end{Highlighting}
\end{Shaded}

\begin{verbatim}
##  mean_speed       stack         gam         pls         pcr          lm 
## 0.000000000 0.002299025 0.002315605 0.002349877 0.002350024 0.002363222 
##         knn       rpart 
## 0.002458646 0.002480831
\end{verbatim}

\hypertarget{conclusion}{%
\section{Conclusion}\label{conclusion}}

Stack a bunch of models for imputing mean point speed. Also do this for
the variance fo the point speed.


\end{document}
